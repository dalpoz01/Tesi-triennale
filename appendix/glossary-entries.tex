% Acronyms
\newacronym[description={\glslink{apig}{Application Program Interface}}]
    {api}{API}{Application Program Interface}

\newacronym[description={\glslink{gbg}{Gigabyte}}]
    {gb}{GB}{Gigabyte}

\newacronym[description={\glslink{mysqlg}{My Structured Query Language}}]
    {mysql}{MySQL}{My Structured Query Language}

\newacronym[description={\glslink{pdfg}{Portable Document Format}}]
    {pdf}{PDF}{Portable Document Format}

\newacronym[description={\glslink{sqlg}{Structured Query Language}}]
    {sql}{SQL}{Structured Query Language}

\newacronym[description={\glslink{umlg}{Unified Modeling Language}}]
    {uml}{UML}{Unified Modeling Language}

\newacronym[description={\glslink{uig}{User Interface}}]
    {ui}{UI}{User Interface}

% Glossary entries
\newglossaryentry{apig} {
    name=\glslink{api}{API},
    text=API,
    sort=api,
    description={in informatica con il termine \emph{Application Programming Interface API} (ing. interfaccia di programmazione di un'applicazione) si indica ogni insieme di procedure disponibili al programmatore, di solito raggruppate a formare un set di strumenti specifici per l'espletamento di un determinato compito all'interno di un certo programma. La finalità è ottenere un'astrazione, di solito tra l'hardware e il programmatore o tra software a basso e quello ad alto livello semplificando così il lavoro di programmazione}
}

\newglossaryentry{gbg} {
    name=\glslink{gb}{GB},
    text=GB,
    sort=gb,
    description={\emph{Gigabyte} (abbreviato in \emph{GB}) è un'unità di misura della quantità di informazioni digitali. Un gigabyte equivale a 1.073.741.824 byte (2^30 byte) o, in termini decimali, a 1.000.000.000 byte (10^9 byte)}
}

\newglossaryentry{mysqlg} {
    name=\glslink{mysql}{MySQL},
    text=MySQL,
    sort=mysql,
    description={\emph{MySQL} è un sistema di gestione di database relazionali (RDBMS) open source basato su SQL (Structured Query Language). È uno dei database più popolari al mondo, ampiamente utilizzato per applicazioni web e software aziendali. MySQL è noto per la sua velocità, affidabilità e facilità d'uso, ed è spesso utilizzato in combinazione con linguaggi di programmazione come PHP, Python e Java per creare applicazioni dinamiche e interattive}
}

\newglossaryentry{pdfg}{
    name=\glslink{pdf}{PDF},
    text=PDF,
    sort=pdf,
    description={\emph{PDF, Portable Document Format} (ing. formato di documento portatile) è un formato di file sviluppato da Adobe Systems per rappresentare documenti in modo indipendente dall'hardware, dal software e dal sistema operativo utilizzati per crearli o visualizzarli. I file PDF possono contenere testo, immagini, grafica vettoriale e altri elementi multimediali, mantenendo la formattazione originale del documento. Il formato PDF è ampiamente utilizzato per la condivisione di documenti elettronici, in quanto garantisce che il contenuto venga visualizzato correttamente su diverse piattaforme e dispositivi}
}

\newglossaryentry{sqlg}{
    name=\glslink{sql}{SQL},
    text=SQL,
    sort=sql,
    description={\emph{SQL, Structured Query Language} (ing. linguaggio strutturato di interrogazione) è un linguaggio standardizzato per la gestione e l'interrogazione dei database relazionali. È utilizzato per definire e manipolare i dati in un database, consentendo operazioni come la creazione, l'aggiornamento, la cancellazione e il recupero di informazioni. SQL è ampiamente utilizzato in applicazioni software e sistemi di gestione di database per interagire con i dati in modo efficiente e strutturato}
}

\newglossaryentry{umlg} {
    name=\glslink{uml}{UML},
    text=UML,
    sort=uml,
    description={in ingegneria del software \emph{UML, Unified Modeling Language} (ing. linguaggio di modellazione unificato) è un linguaggio di modellazione e specifica basato sul paradigma object-oriented. L'\emph{UML} svolge un'importantissima funzione di ``lingua franca'' nella comunità della progettazione e programmazione a oggetti. Gran parte della letteratura di settore usa tale linguaggio per descrivere soluzioni analitiche e progettuali in modo sintetico e comprensibile a un vasto pubblico}
}

\newglossaryentry{uig}{
    name=\glslink{ui}{UI},
    text=UI,
    sort=ui,
    description={\emph{UI, User Interface} (ing. interfaccia utente) è l'interfaccia tra un utente e un sistema informatico. È composta da elementi grafici che consentono all'utente di interagire con il software, come pulsanti, menu, icone e finestre. L'interfaccia utente è progettata per facilitare l'uso del software e migliorare l'esperienza dell'utente}
}
