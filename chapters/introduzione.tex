\chapter{Introduzione}
\label{cap:introduzione}
\intro{In questo capitolo viene presentata l'azienda ospitante, l'idea alla base dello stage
e l'organizzazione del testo.}\\

\section{L'azienda}

Il progetto di \emph{stage} di questa tesi è stato svolto presso l'azienda Bluenext Srl, società di
consulenza informatica fondata nel 2012 a Rimini e operativa su tutto il territorio nazionale.
In particolare, lo \emph{stage} si è svolto presso la sede di Treviso, in una delle filiali distribuite
sul territorio italiano, che in passato operava con il nome di SogeaSoft Srl prima di essere
acquisita da Bluenext nel 2022.

\begin{figure}[!h] 
    \centering 
    \includegraphics[width=0.6\columnwidth]{BluenextSrl_logo_Blu.png} 
    \caption{Logo di Bluenext S.R.L.}
\end{figure}

\subsection{Prodotti dell'azienda}

L'azienda offre soluzioni \emph{software} personalizzate per imprese di varie dimensioni, con
l'obiettivo di migliorare i processi aziendali e aumentare l'efficienza operativa.
Il target principale è rappresentato dalle Piccole e Medie Imprese (PMI) che desiderano
digitalizzare i propri processi e richiedono supporto tecnico specializzato.
L'azienda si occupa dello sviluppo di \emph{software} ERP e il suo prodotto di punta è «SAI»,
un \emph{software} gestionale caratterizzato da una elevata flessibilità che permette di effettuare
integrazioni in qualsiasi direzione, rendendolo estremamente adattabile e personalizzabile
in base alle specifiche esigenze del cliente.

\section{L'idea}

I sistemi di reportistica centralizzata sono strumenti essenziali per le aziende che necessitano di generare,
gestire e distribuire report in modo efficiente. JasperServer è una soluzione ampiamente adottata, 
ma esistono alternative che potrebbero offrire vantaggi in termini di prestazioni, scalabilità, costi e facilità di integrazione. 
Questo progetto mira a valutare diverse alternative a JasperServer, analizzandone le caratteristiche, i punti di forza e le potenziali limitazioni
arrivando a sviluppare un prototipo basato su una delle soluzioni individuate, al fine di testarne l'efficacia in un contesto reale.

\section{Organizzazione del testo}

Il capitolo corrente è l'introduzione del documento, dove si è spiegato brevemente
l'ambito di lavoro e il progetto sul quale si è svolto lo \emph{stage}.
In seguito il documento sarà organizzato con la seguente struttura:

\begin{description}
    \item[{\hyperref[cap:descrizione-stage]{Il secondo capitolo}}] approfondisce la descrizione dello stage, illustrando il contesto aziendale, le tecnologie utilizzate e le sfide affrontate durante il periodo di tirocinio.
    
    \item[{\hyperref[cap:analisi-requisiti]{Il terzo capitolo}}] approfondisce l'analisi dei requisiti del sistema di reportistica, descrivendo il tracciamento dei requisiti.

    \item[{\hyperref[cap:processi-metodologie]{Il quarto capitolo}}]  approfondisce il progetto di \emph{stage}, descrivendo gli obiettivi e le attività svolte, la metodologia di lavoro adottata e le tecnologie analizzate.
    
    \item[{\hyperref[cap:progettazione-codifica]{Il quinto capitolo}}] è dedicato all'analisi delle soluzioni individuate, alla progettazione e alla codifica del prototipo di reportistica basato su una delle alternative a JasperServer individuate.
    
    \item[{\hyperref[cap:prestazioni]{Il sesto capitolo}}] descrive le prestazioni ottenute durante i test effettuati sul prototipo di reportistica sviluppato rispetto a JasperServer.
    
    \item[{\hyperref[cap:conclusioni]{Nel settimo capitolo}}] descrive le conclusioni del lavoro svolto, riassumendo i risultati ottenuti e proponendo possibili sviluppi futuri.
\end{description}

\subsection{Convenzioni tipografiche}

Riguardo la stesura del testo, relativamente al documento sono state adottate le seguenti convenzioni tipografiche:
\begin{itemize}
	\item gli acronimi, le abbreviazioni e i termini ambigui o di uso non comune menzionati vengono definiti nel glossario, situato alla fine del presente documento;
	\item per la prima occorrenza dei termini riportati nel glossario viene utilizzata la seguente nomenclatura: \emph{parola}\glsfirstoccur;
	\item i termini in lingua straniera o facenti parti del gergo tecnico sono evidenziati con il carattere \emph{corsivo};
	\item i comandi, le query, i percorsi di file e il codice inline sono rappresentati tramite il carattere \texttt{monospaziato}.
\end{itemize}
